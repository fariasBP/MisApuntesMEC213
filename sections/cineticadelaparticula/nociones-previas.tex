% !TeX spellcheck = es_PE-SpanishPeru
\subsection*{Nociones previas}
\subsubsection*{Interacciones de los cuerpos}
Se dice que dos cuerpos interactúan cuando ellos actúan entre si, es decir ejercen una acción reciproca. Dicha interacción puede ser de dos tipos:
por contacto o por distancia. Si dos cuerpos se tocan decimos entonces que ellos interactúan por contacto; y si un cuerpo ejerce atracción o repulsión sobre otros que se encuentra a una distancia de el, entonces decimos que la interacción es a distancia. Por ejemplo, patear un balón se califica como una interacción por contacto y la repulsión de dos imanes viene siendo una interacción a distancia.
\subsubsection*{Concepto intuitivo de fuerza}
La fuerza se origina de interacciones entre cuerpos ya se a distancia o por contacto o dicho de una manera coloquial se origina de acciones y reacciones entre dichos cuerpos.
\subsubsection*{Fuerza de la naturaleza}
\subsubsection*{Estado de Movimiento}
Llamaremos estado de movimiento o estado mecánico a la condición física de un cuerpo definida por su velocidad.
Según esta definición dos cuerpos con igual velocidad, poseen el mismo estado de movimiento. Así mismo, dos cuerpos el reposo también tienen el mismo estado de movimiento.
\subsubsection*{Inercia}
\begin{Theorem*} {Inercia}
	La inercia es la tendencia de un cuerpo a permanecer en el mismo estado de movimiento.
\end{Theorem*}
Es decir que si un cuerpo se encuentra en movimiento por inercia esté, continuaré a estar en movimiento, o si esta en reposo esté, continuará a estar en reposo.
\subsubsection*{Masa de un cuerpo}
En fisica clasica, la masa de un cuerpo se define como una magnitud y propiedad física que expresa la inercia o resistencia al cambio de movimiento de un cuerpo. Es decir si cuerpo tiene mas masa, mas fuerza sera necesario para moverlo, o de lo contrario pararlo.
\subsubsection*{Ley de la inercia (primera ley de Newton)}
\begin{Theorem*} {1era ley de Newton - Ley de la inercia}
	Todo cuerpo continua en su estado de reposo o movimiento uniforme en linea recta no muy lejos de las fuerzas impresas a cambiar su posición.
\end{Theorem*}
Es decir que un cuerpo que se halla en reposo continuara en reposo o que se halla en M.R.U. continuara en ese estado a no ser que sobre el actué una fuerza neta externa no nula y modifique su estado mecánico.
Note usted que se habla de un cuerpo en reposo o en movimiento.
