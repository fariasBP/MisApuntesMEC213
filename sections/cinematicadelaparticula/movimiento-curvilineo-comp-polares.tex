\subsection*{Movimiento curvilineo – components polares}
\subsubsection*{Posición}
\begin{Theorem*} {Posición – en función a sus componentes polares}
	El vector posición $\vb{r}$ del punto $P$ en función a sus componentes polares está definida por:
	$$ \vb{r}=r\vb{e_r} $$
\end{Theorem*}
\subsubsection*{Velocidad}
\begin{Theorem*} {Velocidad – en función a sus componentes polares}
	El vector velocidad $\vb{v}$ del punto $P$ en función a sus componentes polares está definida por:
	$$ \vb{v}=v_r\vb{e_r}+v_{\theta}\vb{e_{\theta}} $$
	Donde:
	\begin{flalign*}
		&v_r=\dv{r}{t} \\
		&v_\theta=r\dv{\theta}{t}=r\omega
	\end{flalign*}
\end{Theorem*}
ó tambien expresado de la siguiente manera:
\begin{flalign*}
	&v_r=\dot{r} \\
	&v_\theta=r\dot{\theta}=r\omega
\end{flalign*}
El modulo del vector velocidad en términos de coordenadas polares está dada por:
$$ v=\sqrt{v_r^2+v_\theta^2} $$
\subsubsection*{Aceleración}
\begin{Theorem*} {Aceleración – en función a sus componentes polares}
	El vector aceleración $\vb{a}$ del punto $P$ en función a sus componentes polares está definida por:
	$$ \vb{a}=a_r\vb{e_r}+a_\theta\vb{e_\theta} $$
	Donde:
	\begin{flalign*}
		&a_r=\dv[2]{r}{t}-r\left(\dv{\theta}{t}\right)^2 \\
		&a_\theta=r\dv[2]{\theta}{t}+2\dv{r}{t}\dv{\theta}{t}
	\end{flalign*}
\end{Theorem*}
ó tambien expresado de la siguiente manera:
\begin{flalign*}
	&a_r=\ddot{r}-r\dot{\theta}^2=\ddot{r}-r\omega^2 \\
	&a_\theta=r\ddot{\theta}+2\dot{r}\dot{\theta}=r\alpha+2\dot{r}\omega
\end{flalign*}
El módulo del vector aceleración en términos de coordenadas polares está dada por:
$$ a=\sqrt{a_r^2+a_\theta^2} $$