\subsection*{Movimiento angular}
\subsubsection*{Movimiento angular de una línea en el plano}
\begin{Theorem*} {Posición angular}
	La posición angular de una línea $L$ en un plano respecto a una línea de referencia $L_0$ en el plano puede describirse mediante un ángulo $\theta$.
\end{Theorem*}
\begin{Theorem*} {Velocidad angular}
	La velocidad angular de $L$ respecto $L_0$ en un tiempo $t$ es la razón de cambio de la posición angular $\theta$ con respecto a $t$.
	$$ \omega = \dv{\theta}{t} $$
\end{Theorem*}
\begin{Theorem*} {Aceleración angular}
	La aceleración angular de $L$ respecto a $L_0$ en un tiempo $t$ es la razón de cambio de la velocidad angular $\omega$ con respecto $t$.
	$$ \alpha =\dv{\omega}{t} $$
\end{Theorem*}
\subsubsection*{Ecuación del MAU Y MAUV}
Las ecuaciones del movimiento angular son iguales a las ecuaciones del movimiento rectilíneo.
\begin{itemize}
	\item Movimiento angular uniforme MAU
	\begin{enumerate}
		\item $ \theta = \omega_0 t $
	\end{enumerate}
	\item Movimiento angular uniformemente variado MAUV
	\begin{enumerate}
		\item $\omega=\omega_0+\alpha t$
		\item $ \omega^2=\omega_0^2+2\alpha \theta $
		\item $\theta=\omega_0t=\frac{1}{2}\alpha t^2$
		\item $\theta=\left(\frac{\omega_0+\omega}{2}\right)t$
	\end{enumerate}
\end{itemize}
