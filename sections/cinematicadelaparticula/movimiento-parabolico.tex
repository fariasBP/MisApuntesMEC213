\subsection*{Movimiento curvilíneo - Movimiento parabólico}
Es aquel movimiento en donde su trayectoria describe una parábola.
\subsubsection*{Ecuaciones del movimiento parabólico en función a sus componentes}
Este movimiento a menudo se estudia en funciones a sus componentes x y y, MRU en la componente x y caída libre en la componente y.
\begin{Theorem*} {Ecuaciones del movimiento parabólico en función a sus componentes}
	\begin{itemize}
		\item Componente $x$: describe un MRU, donde $a_x=0$ y es constante, sé cumple que:
		$$ s-s_0=v\cdot t $$
		\item Componente $y$: describe un movimiento en caída libre, donde a y es igual a la aceleración de la gravedad y es constante (siempre que la altura no sea significante), sé cumple que:
		\begin{enumerate}
			\item $v=v_0+gt$
			\item $v2=v_0^2+2g(h-h_0)$
			\item $h-h_0=v_0t+\frac{1}{2}gt^2$
			\item $h-h_0=\left(\frac{v_0+v}{2}\right)\cdot t$
		\end{enumerate}
	\end{itemize}
\end{Theorem*}
\subsubsection*{Ecuación de la trayectoria}
La ecuación de la trayectoria viene de la transformación algebraica de las ecuaciones del movimiento en función a sus componentes mencionada anteriormente.
\begin{Theorem*} {Ecuación de la trayectoria}
	Si la partícula sigue un movimiento parabólico, entonces se cumple que:
\end{Theorem*}
\subsubsection*{Resolución de problemas para movimiento en Caída Libre}
Existe 5 casos básicos de problemas con este tipo de movimiento: \\ 
\begin{itemize}
	\item Cuando la partícula solo baja con un ángulo nulo:
	$$ -y=x\tg(0)-\frac{gx^2}{2v_0^2\cos^2(0)} $$
	$$ y=\frac{gx^2}{2v_0^2} $$
	\item Cuando la partícula solo baja:
	$$ -y=x\tg(-\theta)-\frac{gx^2}{2v_0^2\cos^2(-\theta)} $$
	\item Cuando la partícula sube y baja en un lugar más alto:
	$$ y=x\tg\theta-\frac{gx^2}{2v_0^2\cos^2\theta} $$
	\item Cuando la partícula sube y baja en un lugar más bajo:
	$$ -y=x\tg\theta-\frac{gx^2}{2v_0^2\cos^2\theta} $$
	\item Cuando la partícula sube y baja en la misma posición:
	$$ 0=x\tg\theta-\frac{gx^2}{2v_0^2\cos^2\theta} $$
	$$ x\tg\theta=\frac{gx^2}{2v_0^2\cos^2\theta} $$
\end{itemize}
\subsubsection*{Observaciones del movimiento parabólico}
\begin{itemize}
	\item Cuando el ángulo de inclinación es una variable suele ser útil trabajar con la ecuación de la trayectoria, pero transformada de la siguiente manera:
	$$ y=-\frac{gx^2}{2v_0^2}\tg^2 \theta + x\tg \theta - \frac{gx^2}{2v_0^2} $$
	Donde se usa la identidad para lograr esto.
	$$ 1+\tg^2\theta=\sec^2\theta $$
\end{itemize}
Observaciones cuando la partícula en movimiento parabólico sube y baja en la misma posición
\begin{itemize}
	\item El alcance vertical máximo o altura máxima, está dado por:
	$$ y_{max}=\frac{v_0^2\sen^2\theta}{2g} $$
	\item El alcance horizontal máximo o alcance de lanzamiento, está dado por:
	$$ x_{max}=\frac{v_0^2\sen2\theta}{g} $$
	\item El tiempo de vuelo está determinado por:
	$$ t_{vuelo}=\frac{2v_0\sen\theta}{g} $$
	\item El alcance de lanzamiento cuando la altura sea máxima esta dado por:
	$$ x_{y_{max}}=\frac{v_0^2\sen2\theta}{2g} $$
	\item El ángulo de alcance horizontal máximo se da cuando $\theta=45 \degree$.
	\item El ángulo de alcance vertical máximo se da cuando $\theta=90\degree$.
	\item El ángulo para que la altura máxima sea mayor que el alcance horizontal es: $$ \theta<\tg^{-1}(4) $$ o también aproximadamente $$ \theta<75.96\degree $$Así también ese es el ángulo para que el alcance máximo sea mayor a la altura máxima.
	\item Hay dos ángulos que consiguen el mismo alcance horizontal (excepto para el alcance horizontal máximo), en donde uno siempre es complemento del otro. Es decir, si para un ángulo de $55\degree$ se logra un alcance $x$, entonces se pude lograr el mismo alcance para el ángulo de $35\degree$.
\end{itemize}
