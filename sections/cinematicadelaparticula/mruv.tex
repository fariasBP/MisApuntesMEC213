\subsection*{Movimiento rectilíneo - MRUV}
Considerando la aceleración constante durante toda la trayectoria, se puede modificar las ecuaciones dadas anteriormente.
\begin{itemize}
	\item\textit{\textbf{ La primera ecuación: }} está ecuación la obtenemos modificando la ecuación de la definición de la aceleración:
	\begin{gather*}
		\int_{v_0}^{v}dv = \int_{0}^{t}a dt \\
		v=v_0+at
	\end{gather*}
	\item \textit{\textbf{La segunda ecuación:}} está ecuación la obtenemos modificando la ecuación de la conocida $a(s)$:
	\begin{gather*}
		\int_{v_0}^{v}dv=\int_{0}^{t}ads \\
		v^2=v_0^2+2a(s-s_0)
	\end{gather*}
	\item \textit{\textbf{La tercera ecuación:}} está ecuación la obtenemos modificando la ecuación de la definición de la velocidad y la primera ecuación de MRUV:
	\begin{gather*}
		\int_{s_0}^{s}ds=\int_{0}^{t}(v_0+at)dt \\
		s-s_0=v_0t+\frac{1}{2}at^2
	\end{gather*}
	\item \textit{\textbf{La cuarta ecuación:}} está ecuación la obtenemos modificando la ecuación de la velocidad promedio:
	\begin{gather*}
		\varDelta s = v_{prom}\cdot t \\
		s-s_0=\left(\frac{v_0+v}{2}\right)\cdot t
	\end{gather*}
\end{itemize}
Por lo tanto:
\begin{Theorem*} {Ecuaciones del MRUV}
	En un movimiento rectilíneo, sí la aceleración permanece constante durante toda la trayectoria entonces se cumple las siguientes ecuaciones:
	\begin{enumerate}
		\item $ v=v_0+at $
		\item $ v^2=v_0^2+2a(s-s_0) $
		\item $ s-s_0 = v_0t+\frac{1}{2}at^2 $
		\item $ s-s_0 = \left(\frac{v_0+v}{2}\right) \cdot t $
	\end{enumerate}
\end{Theorem*}