\subsection*{Movimiento curvilíneo – Componentes rectangulares}
\subsubsection*{Posición, velocidad y acleración, y sus componentes rectangulares}
\textit{\textbf{Posición}}
$$ \mathbf{r}=x\mathbf{i}+y\mathbf{j}+z\mathbf{k} $$
\textit{\textbf{Velocidad}} \\
Derivando el vector r con respecto al tiempo se obtiene el vector velocidad y sus componentes:
$$ \mathbf{v}=\dv{\mathbf{r}}{t}=\dv{x}{t}\mathbf{i}+\dv{y}{t}\mathbf{j}+\dv{z}{t}\mathbf{k} $$
$$ \mathbf{v}=\dot{x}\mathbf{i}+\dot{y}\mathbf{j}+\dot{z}\mathbf{k} $$
Expresando el vector velocidad en términos de componentes escalares se obtiene:
$$ \mathbf{v} = v_x\mathbf{i}+v_y\mathbf{j}+v_z\mathbf{k} $$
A partir de esta última, se obtiene ecuaciones escalares que relacionan las componentes de la velocidad y las coordenadas de P:
\begin{flalign*}
	&v_x=\dot{x}=\dv{x}{t} & &v_y=\dot{y}=\dv{y}{t} & &v_z=\dot{z}=\dv{z}{t}
\end{flalign*}
Y con las componentes escalares se obtiene el módulo del vector velocidad:
$$ v=\sqrt{v_x^2+v_y^2+v_z^2} $$
\textit{\textbf{Aceleración}} \\
Derivando el vector v con respecto al tiempo se obtiene el vector aceleración y sus componentes:
$$ \mathbf{a}=\dv{\mathbf{v}}{t}=\dv{v_x}{t}\mathbf{i}+\dv{v_y}{t}\mathbf{j}+\dv{v_z}{t}\mathbf{k} $$
$$ \mathbf{a}=\ddot{x}\mathbf{i}+\ddot{y}\mathbf{j}+\ddot{z}\mathbf{k} $$
Expresando el vector aceleración en términos de componentes escalares se obtiene:
$$ \mathbf{a}=a_x\mathbf{i}+a_y\mathbf{j}+a_z\mathbf{k} $$
A partir de esta última, se obtiene ecuaciones escalares que relacionan las componentes de la aceleración y las coordenadas de P:
\begin{flalign*}
	&v_x=\ddot{x}=\dv{v_x}{t} & &v_y=\ddot{y}=\dv{v_y}{t} & &v_z=\ddot{z}=\dv{v_z}{t}
\end{flalign*}
Y con las componentes escalares se obtiene el módulo del vector aceleración:
$$ a=\sqrt{a_x^2+a_y^2+a_z^2} $$
En conclusión:
\begin{Theorem*} {Posición, velocidad y aceleración en función a sus componentes rectangulares}
	Sea un punto P en movimiento con velocidad y aceleración que sigue una trayectoria curvilínea y en donde: r es el vector posición, v el vector velocidad y a el vector aceleración; entonces sus componentes rectangulares son:
	$$ \mathbf{r}=x\mathbf{i}+y\mathbf{j}+z\mathbf{k} $$
	$$ \mathbf{v} = v_x\mathbf{i}+v_y\mathbf{j}+v_z\mathbf{k} $$
	$$ \mathbf{a}=a_x\mathbf{i}+a_y\mathbf{j}+a_z\mathbf{k} $$
\end{Theorem*}

