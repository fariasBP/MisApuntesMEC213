\subsection*{Movimiento relativa}
\subsubsection*{Posición relativa}
\begin{Theorem*} {Posición relativa de un punhto con respecto a otro}
	Sean $\vb{r}_A$ y $\vb{r}_B$ las posiciones de dos puntos $A$ y $B$ respecto al origen $O$ de un marco de referencia dado. La posición de $B$ es igual a la posición de $A$ más la posición de de $B$ respecto a $A$:
	$$ \vb{r}_B=\vb{r}_A+\vb{r}_{B/A} $$
\end{Theorem*}
Es decir, la posición de $B$ con respecto a $A$ es igual a la diferencia de ambas posiciones respectivamente:
$$ \vb{r}_{B/A}=\vb{r}_B-\vb{r}_A $$
\subsubsection*{Velocidad relativa}
Si se toma la derivada con respecto al tiempo de la ecuación de posición relativa se obtiene la ecuación de velocidad relativa, entonces:
\begin{Theorem*} {Velocidad relativa de un punto con respecto a otro}
	La velocidad de $B$ respecto a $O$ es igual a la velocidad de $A$ respecto a $O$ mas la velocidad $\vb{v}_{B/A}$ de $A$ respecto a $B$:
	$$ \vb{v}_B=\vb{v}_A+\vb{v}_{B/A} $$
\end{Theorem*}
Es decir, la velocidad de $B$ respecto a $A$ es igual a la diferencia de ambas velocidades respectivamente:
$$ \vb{v}_{B/A}=\vb{v}_B+\vb{v}_A $$
\subsubsection*{Aceleración relativa}
Si se toma la derivada con respecto al tiempo de la ecuación velocidad relativa se obtiene la ecuación de aceleración relativa, entonces:
\begin{Theorem*} {Aceleración relativa de un punto respecto a otro}
	La aceleración de $B$ respecto a $O$ es igual a la aceleración de $A$ respecto a $O$ mas la velocidad $\vb{a}_{B/A}$ de $A$ respecto a $B$:
	$$ \vb{a}_B=\vb{a}_A+\vb{a}_{B/A} $$
\end{Theorem*}
Es decir, la aceleración de $B$ con respecto a $A$ es igual a la diferencia de ambas aceleraciones respectivamente:
$$ \vb{a}_{B/A}=\vb{a}_B-\vb{a}_A $$