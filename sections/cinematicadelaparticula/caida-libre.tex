\subsection*{Movimiento rectilíneo - Caída libre de una partícula}
La caída libre es un movimiento rectilíneo que depende de la gravedad si y solo si esté en un sistema cerrado, entonces este movimiento tiene la característica de que la aceleración sea igual a la gravedad; además si este movimiento ocurre a una altura insignificante (altura pequeña) donde el valor de la gravedad no tenga una variación notable, entonces podemos tomar a este valor de la gravedad como constante, esto conlleva que podemos usar las ecuaciones del MRUV.
\begin{Theorem*} {Ecuaciones del MRUV en Caída Libre}
	En un movimiento en Caída Libre, donde la altura y la variación de gravedad son insignificantes entonces consideramos a la gravedad como constante, a consecuencia se cumplirá que:
	\begin{enumerate}
		\item $v=v_0+gt$
		\item $v^2=v_0^2+2g(h-h_0)$
		\item $h-h_0=v_0t+\frac{1}{2}gt^2$
		\item $h-h_0=\left(\frac{v_0+v}{2}\right)\cdot t$
	\end{enumerate}
\end{Theorem*}
\subsubsection*{Resolución de problemas para movimiento en Caída Libre}
Los problemas de movimiento de caída libre lo tomaremos como un movimiento rectilíneo uniformemente variado, en donde la aceleración será la gravedad y lo trabajaremos como una constante. Existe 5 casos básicos de problemas con este tipo de movimiento:
\begin{enumerate}
	\item Cuando la partícula solo baja
	\begin{align*}
		&v=v_0+gt \\
		&v^2=v_0^2+2g(h-h_0) \\
		&h-h_0=v_0t+\frac{1}{2}gt^2 \\
		&h-h_0=\left(\frac{v_0+v}{2}\right)\cdot t
	\end{align*}
	\item Cuando la partícula solo sube
	\begin{align*}
		&v=v_0-gt \\
		&v^2=v_0^2-2g(h-h_0) \\
		&h-h_0=v_0t-\frac{1}{2}gt^2 \\
		&h-h_0=\left(\frac{v_0+v}{2}\right)\cdot t
	\end{align*}
	\item Cuando la partícula sube y baja en un lugar más alto
	\begin{align*}
		&v=v_0-gt \\
		&v^2=v_0^2-2g(h-h_0) \\
		&h-h_0=v_0t-\frac{1}{2}gt^2 \\
		&h-h_0=\left(\frac{v_0+v}{2}\right)\cdot t
	\end{align*}
	\item Cuando la partícula sube y baja en un lugar más bajo
	\begin{align*}
		&-v=v_0-gt \\
		&(-v)^2=v^2=v_0^2-2g(h-h_0) \\
		&(-h)-h_0=v_0t-\frac{1}{2}gt^2 \\
		&(-h)-h_0=\left(\frac{v_0-v}{2}\right)\cdot t
	\end{align*}
	\item Cuando la partícula sube y baja en la misma posición
	\begin{align*}
		&v=v_0-gt \\
		&v^2=v_0^2-2g(0) \Rightarrow v^2=v_0^2 \Rightarrow v=v_0 \\
		&0=v_0t-\frac{1}{2}gt^2 \Rightarrow v_0=\frac{1}{2}gt \\
		&0=\left(\frac{v_0-v}{2}\right)\cdot t \Rightarrow v_0=v
	\end{align*}
\end{enumerate}
\textit{\textbf{Observaciones:}}
\begin{itemize}
	\item Lo que se usa en realidad es el plano cartesiano para determinar el signo del las variables:
	\item Note el caso 3 y 4, en donde en vez de realizar dos resoluciones, uno hacia arriba y otro hacia abajo; se realiza una resolución directo al punto final.
	\item El punto mas alto que alcanza la partícula en este movimiento la velocidad es nula:
	\item Si las alturas de subida y  bajada son las mismas (caso 5), entonces se cumple que: $v_A=v_E\Rightarrow v_{iniciar} = v_{final}$ y también $t_{subida}=t_{bajada}$.
	
\end{itemize}