\subsection*{Movimiento dependiente y Poleas}
En algunos tipos de problemas el movimiento de una partícula dependerá del movimiento de otra partícula, por ejemplo, en el grafico siguiente se muestra la partícula $A$ (representado por un bloque que se desprecia sus dimensiones) que se mueve en un plano inclinado, este movimiento provoca el movimiento de la partícula $B$; entonces se puede decir que el bloque $B$ depende de $A$. \\
Donde la relación entre desplazamiento, velocidad y aceleración se obtiene de la siguiente manera:
$$ l_T=s_A+l_{CD}+s_B $$
\begin{itemize}
	\item Se diferencia
	\begin{flalign*}
		&\dv{l_t}{t}=\dv{s_A}{t}+\dv{l_{CD}}{t}+\dv{s_B}{t} \\
		&0=v_A+0+v_B\\
		&v_A=-v_B
	\end{flalign*}
	\item Se vuelve a diferenciar
	\begin{flalign*}
		&\dv{v_A}{t}=-\dv{v_B}{t} \\
		&a_A=-a_B
	\end{flalign*}
\end{itemize}
\subsubsection*{Procedimiento para el análisis}
\begin{itemize}
	\item Se establece un origen en un plano de referencia para cada coordenada de posición.
	No es necesario que el origen sea el mismo para cada una de las coordenadas, pero es importante que cada eje tenga un mismo origen.
	\item Con geometría y/o trigonometría, se debe relacionar las coordenadas de posición con la longitud total de la cuerda l, o con alguna porción de la misma.
	\item En la mayoría de los casos (por no decir todos) se excluye las porciones de la cuerda que no cambian de longitud a medida que las partículas se muevan, por ejemplo, las porciones que se encuentran en el arco enrollados sobre la polea:
	\item Si el problema implica un sistema de dos o mas cuerdas enrolladas alrededor de las poleas entonces la posición de un punto en una cuerda debe relacionarse con posición de un punto en otra cuerda.
	Además, se deben escribir ecuaciones distintas para cada cuerda del sistema de cuerdas. (Ver ejemplo 2).
	\item Se deriva la posición para obtener la velocidad y así nuevamente para obtener la aceleración.
\end{itemize}
\subsubsection*{Poleas y movimiento dependiente absoluto}
Otra forma alterna para operar con poleas es la siguiente:
\begin{itemize}
	\item Para poleas fijas
	\item Para poleas móviles
\end{itemize}


