\subsection*{Movimiento circular}
\begin{Theorem*} {Movimiento circular}
	El punto $P$ describe un movimiento circular sí y solo sí el radio de curvatura permanece constante durante toda la trayectoria:
	$$ \rho=\text{ctte}\Rightarrow\rho=R $$
\end{Theorem*}
En un movimiento donde la trayectoria es curvilínea el radio de curvatura cambia durante toda la trayectoria, sin embargo, si este valor se mantiene constante, entonces decimos que la trayectoria describe un círculo. Es decir, sí el radio de curvatura es constante, el movimiento es circular y al radio de curvatura ahora lo llamaremos radio denotada por $R$. Entonces tambien se podrá trabajar de la misma manera que el movimiento angular. \\
y también se puede trabajar de la misma manera que con el movimiento curvilíneo en función a sus componentes normal y tangencial: \\
Por lo tanto, en movimiento circular se puede trabajar con los mismos procedimientos del movimiento angular y movimiento curvilíneo normal-tangencial.
Relación entre el movimiento angular y el movimiento normal-tangencial. \\
\textit{\textbf{Relación movimiento angular y movimiento curvilíneo normal-tangencial}}
\begin{Theorem*} {Relación movimiento angular y movimiento normal-tangencia}
	Sea el punto $P$ que describe un movimiento circular, se cumple que:
	\begin{flalign*}
		&s=R\theta \\
		&v=R \ \dv{\theta}{t}=r\omega \\
		&a_t=r \ \dv{\omega}{t} = R\alpha \\
		&a_n=\frac{v^2}{R}=R\omega^2
	\end{flalign*}
\end{Theorem*}