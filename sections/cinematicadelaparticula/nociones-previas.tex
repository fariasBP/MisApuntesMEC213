\subsection{Nociones Previasss}
\subsubsection{Cantidades Vectoriales}
\begin{Theorem*}{Velocidad}
	La posición es la ubicación de un punto $ P $ en relación con un sistema coordenado especifico, o marco de referencia, con origen en el punto $ O $ y representado por el vector de posición denominado $ \mathbf{r} $ o también $ \vec{r} $.
	$$ \mathbf{r}=\mathbf{r}(t) $$
\end{Theorem*}
\begin{Theorem*}{Velocidad}
	La velocidad de un punto P relativa a O es la razón de cambio de posición con respecto al tiempo y es representado por el vector velocidad denominado $ \mathbf{v} $ o también $ \vec{v} $.
	$$ \mathbf{v}=\frac{d\mathbf{r}}{dt} $$
\end{Theorem*}
Además es importante tener en cuenta lo siguiente:
\begin{Theorem*}{Velocidad tangente a la trayectoria}
	En cualquier movimiento, la velocidad siempre es tangente a la trayectoria
\end{Theorem*}
\begin{Theorem*}{Aceleración}
	La aceleración de un punto $ P $ relativa a $ O $ es la razón de cambio de velocidad con respecto al tiempo y es representado por el vector aceleración denominado $ \mathbf{a} $ o también $ \vec{a} $.
	$$ \mathbf{a}=\frac{d\mathbf{v}}{dt} $$
\end{Theorem*}