\usepackage[utf8]{inputenc}
\usepackage{amsmath}
\usepackage{amssymb}
\usepackage{amsfonts}
\usepackage[left=2cm,right=2cm,top=2cm,bottom=2cm]{geometry}
\usepackage{lipsum}
\usepackage{xcolor}
\usepackage{sectsty}
\usepackage{multicol} % para columnas
\usepackage{tcolorbox} % para cajas
\usepackage[all]{xy} % para matrices con flecha (usada para el metodo aspa entre otros)

\usepackage{physics} % para derivadas variables fisicas etc.

% definiendo importador de imagenes
\usepackage{float}
\usepackage{graphicx} % para importar imagenes
\usepackage{cancel} %para cancelar, tachar expresiones (util para la notacion de simplificar)
\usepackage{gensymb} % para grado ej./ 50°

% para graficas y dibujos
\usepackage{tikz,pgfplots}